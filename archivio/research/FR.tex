\documentclass[pre,aps]{revtex4}

\usepackage{graphicx}
\usepackage{amsmath,amsfonts}

%%%%%%%%%%%%%%%%%%%%%%%%%%%%%%%%%%%%%%%%%%%%%%%%%%%%%%%%%%%%%%%%%%%%%%%%%%%%
%%%%%%%%%%%%%%%%%%%%%            COMANDI              %%%%%%%%%%%%%%%%%%%%%%
%%%%%%%%%%%%%%%%%%%%%%%%%%%%%%%%%%%%%%%%%%%%%%%%%%%%%%%%%%%%%%%%%%%%%%%%%%%%

\newcommand{\beq}{\begin{equation}}
\newcommand{\eeq}{\end{equation}}
\newcommand{\bea}{\begin{eqnarray}}
\newcommand{\eea}{\end{eqnarray}}
\newcommand{\re}{\text{Re}}
\newcommand{\im}{\text{Im}}
\newcommand{\ie}{i.e. }
\newcommand{\eg}{e.g. }
\newcommand{\T}{{\rm T}}
\newcommand{\su}{\sigma^{(1)}}
\newcommand{\sd}{\sigma^{(2)}}
\newcommand{\st}{\sigma^{(3)}}

%%%%%%%%%%%%%%%%%%%%%%%%%%%%%%%%%%%%%%%%%%%%%%%%%%%%%%%%%%%%%%%%%%%%%%%%%%%%%
%%%%%%%%%%%%%%%%%%%%%%         SIMBOLI VARI           %%%%%%%%%%%%%%%%%%%%%%%
%%%%%%%%%%%%%%%%%%%%%%%%%%%%%%%%%%%%%%%%%%%%%%%%%%%%%%%%%%%%%%%%%%%%%%%%%%%%%

\let\a=\alpha \let\b=\beta  \let\g=\gamma  \let\d=\delta \let\e=\varepsilon
\let\z=\zeta  \let\h=\eta   \let\th=\theta \let\k=\kappa \let\l=\lambda
\let\m=\mu    \let\n=\nu    \let\x=\xi     \let\p=\pi    \let\r=\rho
\let\s=\sigma \let\t=\tau   \let\f=\varphi \let\c=\chi
\let\ps=\psi  \let\y=\upsilon \let\o=\omega\let\si=\varsigma
\let\G=\Gamma \let\D=\Delta  \let\Th=\Theta\let\L=\Lambda \let\X=\Xi
\let\P=\Pi    \let\Si=\Sigma \let\F=\Phi    \let\Ps=\Psi
\let\O=\Omega \let\Y=\Upsilon

\let\io=\infty
\let\ul=\underline

%%%%%%%%%%%%%%%%%%%%%%%%%%%%%%%%%%%%%%%%%%%%%%%%%%%%%%%%%%%%%%%%%%%%%%%%%%%%%
%%%%%%%%%%%%%%%%%%%%%%%%%%%%%%%%%%%%%%%%%%%%%%%%%%%%%%%%%%%%%%%%%%%%%%%%%%%%%

\begin{document}

\title{The fluctuation theorem beyond linear response theory \\ and
an extension to systems with slow dynamics}

\maketitle

{\it Disclaimer: this text was written in 2006: references are not updated and all developments
in the field after 2006 are not included}

\bigskip

I started this research at the beginning of my PhD under the supervision
of Giancarlo Ruocco and in collaboration with L.Angelani. 
Initially we performed molecular dynamics simulations.
The theoretical results that followed were motivated by these simulations, and were obtained 
in a series of collaborations with F.Bonetto, L.F.Cugliandolo, G.Gallavotti, A.Giuliani and J.Kurchan.
In the following I will briefly recall the statement of the
Gallavotti-Cohen fluctuation theorem and the reasons why it
attracted much interest in the last decade. 
Then I will describe our contribution to this subject.

\subsection{The chaotic hypothesis and the fluctuation relation}

The ergodic hypothesis plays a central role in equilibrium statistical mechanics.
Although very few systems can be proven to be ergodic, most systems behave effectively
as if they were ergodic. In out of equilibrium statistical physics, it is believed that
Anosov systems (a particular class of smooth chaotic systems) 
play the role of ergodic systems in equilibrium statistical physics.

\vskip10pt
\noindent
{\it Anosov systems and the fluctuation theorem} - 
The fluctuation theorem concerns fluctuations of phase space
contraction in reversible hyperbolic (Anosov) systems. If time evolution is
described by a differential equation on phase space $M$: $\dot
x=X(x),\, x\in M$,
or by a map $S: x\rightarrow S(x)$ of $M$ one defines the {\it phase space
contraction} as, respectively, $\sigma(x)=-{\rm \,div\,} X(x)$ or
$\sigma(x)=-\log |\det\partial_x S(x)|$. Reversibility means that there is a
metric preserving map $I$ of $M$ such that $IS=S^{-1}I$ if $S$ is the
time evolution over a certain time $t$ ({\it e.g.} $t=1$).
{\it If the system is Anosov}, that is if $M$ is compact and $S$ is smooth
and uniformly hyperbolic, see \cite{Ga99,GC95a,GC95b,Ga95c,Ru99}, 
the points $x$ will have a well defined
statistics (or Sinai-Ruelle-Bowen distribution) $\mu_{srb}$, \cite{Ru99}, 
\ie almost all points 
with respect to the volume measure will evolve in time
so that {\it all smooth observables} will have a well defined average
equal to the integral over the SRB distribution. 
The SRB distribution plays in nonequilibrium statistical mechanics the role
that the Boltzmann distribution plays in equilibrium.
Hence, in particular,
the time average of the function $\sigma(x)$ 
will be asymptotically given by the spatial average with respect to 
the SRB distribution. In the case of discrete time maps:
\beq
\label{1.1}
\sigma_+\equiv\lim_{\t\to\io} \frac1\t \sum_{j=0}^{\tau-1} \sigma(S^j(x))
=\int_M \sigma\;d\mu_{srb} \equiv \langle\; \sigma \;\rangle_{srb}\eeq
and an analogous relation holds in the continuous time case.
A general result of Ruelle \cite{Ru99} is that $\s_+ \geq 0$. 
Ruelle's theorem can be thought as a ``Boltzmann H theorem'' for nonequilibrium
statistical mechanics. 
If $\s_+=0$ the system conserves the phase space volume and the SRB distribution
reduces to the volume measure, \ie the system is in equilibrium.
In the dissipative case $\sigma_+ > 0$, let (in the discrete time case):
\beq
\label{FTC1}
p(x)=\frac{1}{\tau \sigma_+} \sum_{j=0}^{\tau-1} \sigma(S^j(x))
\eeq
(an analogous definition is given 
in the continuous time case).
The function $p(x)$ will have average $\langle\; p\;\rangle_{srb} = 1$
and distribution $\pi_\tau(dp)$ such that
\beq
\label{FTC2}
\pi_\tau(\{p\in \Delta\})= e^{\tau \max_{p\in\Delta}\zeta_\infty(p)+ O(1)}\;,
\eeq
where the correction at the exponent is $O(1)$ w.r.t. $\t$ as $\t\to\io$,
\ie the {\it large deviation function} $\z_\io(p)$ is well defined.
The {\it Gallavotti-Cohen fluctuation theorem} states that the following 
fluctuation relation (FR) holds:
\beq
\label{FTC3}
\zeta_\infty(p)=\zeta_\infty(-p)+p \sigma_+ \qquad{\rm for\ all}\ |p|<p^*
\eeq
where $\infty> p^*\ge1$ is a suitable (model dependent) constant, defined
by $\z_\io(p) = -\io$ for $|p|> p^*$. The fluctuation relation was discovered 
in a numerical simulation in \cite{ECM93} and formulated as a theorem for 
Anosov systems in \cite{GC95a}.

\vskip10pt

\noindent
{\it The chaotic hypothesis and physical implications of the FR} - 
Hyperbolicity is a paradigm 
for disordered systems similar to the
small oscillations paradigm used for ordered motions: it does not
hold exactly in essentially all the physically interesting
systems. The {\it chaotic hypothesis} \cite{Ga99,GC95a,GC95b,Ga04,Ru04}
is that nevertheless
one can assume that chaotic motions 
(in the sense of motions with at least one positive Lyapunov exponent) 
exhibit some average properties of truly
hyperbolic motions. This hypothesis is a natural generalization of 
the ergodic hypothesis,
\ie of the assumption that systems of many particles at equilibrium
are well 
described on average by the microcanonical (or by the Gibbs) 
distribution, even if they are not
really (or they are not proven to be) ergodic.
A consequence of the chaotic hypothesis is that
(dissipative) deterministic chaotic reversible 
motions
should have fluctuations of phase space contraction verifying Eq.~\ref{FTC3}.

One interesting example of such motions is given by a system of 
$N$ interacting particles
in $d$ dimensions subjected to nonconservative forces and kept in a
stationary state by a {\it reversible mechanical thermostat}.
It will be defined by a differential equation $\dot x=X_E(x)$ where 
$x=(\dot{\ul q},\ul q) \in R^{2dN} \equiv M$ ({\it phase space})
and
\beq
\label{INTRO1}
m \ddot{\ul q} = \ul f(\ul q) + \ul g_E (\ul q) - \ul 
\theta_E (\dot{\ul q},\ul q)
\eeq
where $m$ is the mass of the particles, $\ul f(\ul q)$ describes 
the internal (conservative)
forces between the particles and $\ul g_E (\ul q)$ represents 
the nonconservative
``external'' force acting on the system. Finally, 
$\ul \theta_E (\dot{\ul q},\ul q)$ is a
mechanical force that prevents the system to acquire energy indefinitely: 
this is why we shall
call it a {\it mechanical thermostat}.
Systems belonging to this class are frequently 
used as microscopic models to describe
nonequilibrium stationary states induced by the application of a driving force
(temperature or velocity gradients, electric fields, etc.) on a
fluid system in contact with a thermal bath, \cite{Ru04}. 
In this context, the phase space contraction
rate $\sigma(x)$ has been identified (setting $k_B=1$) 
with the {\it entropy production rate} \cite{ECM93,Ga99,Ga04,Ru04},
and the fluctuation relation has been successfully tested in 
several numerical simulations
\cite{ECM93,BGG97,BCL98,BPV98,GP99,GRS04,ZRA04a} and
experiments on granular materials and on turbulent
flows \cite{CL98,GGK01,CGHLP03,FM04}.
Having defined the notion of entropy production rate one can 
define a ``duality'' between 
fluxes $\ul J = \{ J_i \}$ and forces $\ul E= \{ E_i \}$ using $\sigma(x)$ as a 
``Lagrangian'' \cite{Ga04}:
\beq
\label{Jdef}
J_i(\ul E,x) = \frac{\partial \sigma(x)}{\partial E_i}
\eeq
In the limit $\ul E \rightarrow \ul 0$, {\it i.e.} close to equilibrium, 
the fluctuation relation
leads to Onsager's reciprocity and to Green-Kubo's formulas for 
transport coefficients
\cite{GR97,Ga96a,Ga96b}:
\beq
\label{GK}
\mu_{ij} \equiv \lim_{\ul E \rightarrow \ul 0} \frac{\langle J_i 
\rangle_{\ul E}}{E_j} = 
\int_0^\infty dt \ \langle J_i(t) J_j(0) \rangle_{\ul E=\ul 0}
\eeq
The reasons why the fluctuation relation attracted lot of interest is that
it is a {\it universal} relation (\ie it contains no model dependent parameters)
that reduces to well known universal relations (the Green-Kubo relations) 
in the linear response regime.
Moreover the chaotic hypothesis, that implies the validity of the fluctuation
relation for reversible systems, is very powerful: it provides an explicit expression
for the invariant measure describing nonequilibrium stationary states of a large
class of systems. Thus it is interesting to explore its validity. 

\vskip10pt
\noindent
{\it Numerical verification of the chaotic hypothesis} -
The simplest check of the applicability of the chaotic hypothesis
is a verification of the fluctuation relation: of course even if the
check has a positive result this will not ``prove'' the hypothesis but
it will at least add confidence to it. A rather stringent test of the
fluctuation relation would be a check which
{\it cannot be reduced to a kind of Green-Kubo relation}; this  requires
at least one of the two following conditions to be satisfied \cite{ZRA04a,GGZ05}:
\begin{enumerate}
\item the distribution $\pi_\tau(p)$ is distinguishable from a Gaussian, or
\item deviations from the linear response theory, \ie
deviations from the Green-Kubo relation, are observed.
\end{enumerate}
This is very hard to obtain in numerical simulations of 
Eq.~\ref{INTRO1} for the following
reasons:
\begin{enumerate}
\item to observe deviations from linearity in Eq.~\ref{GK} one has to
apply very large forces $E$, then $\sigma_+$ is very large and 
it becomes very difficult
to observe the negative values of $p(x)$ that are needed to 
compute $\zeta_\infty(-p)$ in Eq.~\ref{FTC3};
\item deviations from Gaussianity in $\pi_\tau(p)$ 
are observed only for values of $p$ that
differ significantly (of the order of twice the variance) from $1$ and,
again, it is very difficult
to observe such values of $p$.
\end{enumerate}
Due to the limited computational resources available in the past decade,
all numerical computations that can be found in the literature
on systems described by Eq.~\ref{INTRO1} found that the measured
distribution $\pi_\tau(p)$ could not be 
distinguished from a Gaussian distribution
in the interval of $p$ accessible to the numerical experiment 
\cite{ECM93,BGG97,BCL98,ZRA04a}. 

\vskip10pt

The purpose of our work (in collaboration with G.~Gallavotti and
A.~Giuliani) was to test the fluctuation relation, 
in a numerical simulation
of a system described by Eq.~\ref{INTRO1}
(namely, a system of $N=8$ --in $d=2$ dimensions--
and of $N=20$ --in $d=3$-- Lennard--Jones-like particles perturbed
by a constant force), for large 
applied force when deviations from linearity can be observed, and
the distribution $\pi_\tau(p)$ is appreciably non-Gaussian.
This was made possible by the large computational power 
(two clusters of 18 and 32 biprocessor units)
avalaible in the INFM-Soft research center. However,
it was still very difficult to reach values of $\tau$ which
can be confidently regarded as ``close'' to the asymptotic limit 
$\tau \rightarrow \infty$;
thus to interpret our results we developed a theory of the $o(1)$ 
corrections to the function
$\zeta_\infty(p)$ in order to extract the limiting function 
$\zeta_\infty(p)$ from the
numerical data. 
In order to compute the finite time corrections, 
we proposed an algorithm which allows to reconstruct the asymptotic 
distribution function from measurable quantities at finite time, within
a given precision \cite{GGZ05}. Our theory of the corrections relies on the 
symbolic representation of the chaotic dynamics, then it is 
applicable if one accepts the Chaotic Hypothesis.
Taking into account the latter finite time corrections, 
we successfully tested the fluctuation relation for 
non--Gaussian distributions and beyond the
linear response theory.

Our interpretation of the numerical results is that the 
chaotic hypothesis can be applied to these
systems, also very far from equilibrium, and in particular
the fluctuation relation is 
verified even in regions where its predictions measurably differ 
from those of linear response theory.

A big open problem we are left with is trying to understand 
how the fluctuation relation is modified for values
of the driving force so high that the attractive set is no more dense in phase
space. It is expected, \cite{BG97,Ga99b},
that in such a case 
$\z_\io(p)-\z_\io(-p)$ is still linear, but the slope is $X\s_+$, with 
$X$ given by the ratio of the dimension of the attractive set and
of that of the whole phase space. An estimate of such quantity can be given 
via the number of negative pairs of exponents in the Lyapunov spectrum.
Unfortunately negative pairs begin to appear in the Lyapunov spectrum
only for values of the external force so high that no negative fluctuations
are observable anymore.
We hope that future work will address this point. A recently proposed
algorithm to measure large deviations \cite{GKP05} may be useful
in this perspective.


\subsection{Extension of the fluctuation theorem to driven glassy systems}

The fluctuation theorem has been extended to a Langevin equation by Kurchan \cite{Ku98} and
to generic Markov processes by Lebowitz and Spohn \cite{LS99}.
Thus, at present, the fluctuation relation is believed to be a very general relation that characterizes
the fluctuations of the entropy production in out of equilibrium stationary states; moreover,
a close connection between the fluctuation relation and the definition of a 
{\it nonequilibrium temperature} has been conjectured \cite{Ga04,CR04,SCM04,Se98}.

To exploit this connection, one should consider systems such that the nonequilibrium temperature
is not trivially equal to the ambient temperature; 
for a system coupled to a single thermal bath, this happens if 
{\it (i)} the thermal bath has temperature $T$, but the system is not able to equilibrate with
the bath, and/or if {\it (ii)} the bath itself
is not at equilibrium, \ie it does not verify the fluctuation-dissipation relation \cite{Cu02}.
The first situation happens, for example, in glassy systems, that never reach equilibrium with
the thermal bath; the second situation is realized if one consider the diffusion of a Brownian
particle in a complex medium (\eg a glass, or a granular) \cite{PM04,Po04,AG04}.
In this case the medium, which acts as
a thermal bath with respect to the Brownian particle, is itself out of equilibrium.
The two situations are closely related as, at least at the mean field level, the problem of 
glassy dynamics can be mapped in the problem of a single ``effective'' degree of freedom moving 
in an out of equilibrium environment \cite{Cu02,CK99}, 
and thus situations {\it (i)} and {\it (ii)} are described
by the same kind of equation, namely a Langevin equation for a single degree of freedom coupled
to a complex bath, \ie a bath that does not verify the fluctuation-dissipation theorem.
Finally, note that the simplest situation where the nonequilibrium temperature is not trivial
happens when the system is in contact with a certain number of thermal baths at different 
temperature; the latter can also be seen as a model for a ``complex bath'' \cite{CK99}.

Our aim was then to discuss the validity of the fluctuation relation for the simplest
possible system realizing one of the situations discussed above, namely a Brownian particle subject
to a driving force and to a confining potential, and coupled to either a complex (out of equilibrium)
bath or to $N$ different baths with different temperature.
We found that:
\begin{enumerate}
\item the PDF of the entropy production $\s(t) = W(t)/\Th$, where $W(t)$ is the power dissipated 
by the external force and $\Th$ is a free parameter with the dimensions of a temperature,
does not satisfy the fluctuation relation exactly: 
if the bath acts on a single time scale 
the fluctuation relation is verified approximately if $\Th$ is
suitably chosen; if the bath acts on different time scales, 
the function $[\z_\io(p)-\z_\io(-p)]/\s_+$ may have different slopes corresponding to the
different temperatures at small $p$ and at large $p$;
\item if a suitable (frequency dependent)
{\it effective temperature} is defined as a property of the bath \cite{Cu02}, 
and if the entropy production rate is defined as the integral over all frequencies $\o$ of the power
injected by the driving force at frequency $\o$ divided by the effective temperature at the same 
frequency, the fluctuation relation holds for the complex bath we considered;
\item The modified fluctuation relation is related, in the linear regime, to
  the modified Green-Kubo relations in which the effective temperature enters
  instead of the ambient temperature.
\end{enumerate}
Models like the one we discussed have been recently investigated \cite{AG04,PM04,Po04}
to describe the dynamics of Brownian particles in complex media;
Brownian particles are often used as probes in order to study the properties 
of the medium (\eg in Dynamic Light Scattering or Diffusing Wave Spectroscopy experiments).
Moreover, confining potentials for Brownian particles can be generated using laser beams 
\cite{Gr03} and experiments on the fluctuations of the power dissipated in such systems
are currently being performed \cite{AG04,WSMSE02}.

This part of the work is done in collaboration with F.~Bonetto, L.~Cugliandolo and J.~Kurchan 
and the results have been published in \cite{CKZ?}. The analytical results are in good 
agreement with the numerical simulations on Lennard--Jones systems that motivated them, 
performed in collaboration with G.Ruocco and L.Angelani and published in \cite{ZRA04b}.


\begin{thebibliography}{99}




%%Fluctuation theorem

\bibitem{Ga99} G.~Gallavotti, {\it Statistical mechanics. A short treatise}
(Springer Verlag, Berlin, 1999).

\bibitem{GC95a} G.~Gallavotti and E.G.D.~Cohen,
%{\it Dynamical ensembles in nonequilibrium statistical mechanics},
Phys.~Rev.~Lett. {\bf 74}, 2694--2697 (1995).

\bibitem{GC95b} G.~Gallavotti, E.~G.~D.~Cohen, 
%{\it Dynamical
%ensembles in stationary states}, 
J.~Stat.~Phys. {\bf 80}, 931--970 (1995).

\bibitem{Ga95c} G.~Gallavotti, 
%{\it Reversible Anosov maps and large 
%deviations},
Mathematical Physics Electronic Journal, MPEJ, {\bf 1}, 1--12 (1995).

\bibitem{Ru99} D.~Ruelle, 
%{\it Smooth dynamics and new theoretical ideas 
%in non-equilibrium statistical mechanics}, 
J.~Stat.~Phys. {\bf 95}, 393--468 (1999).

\bibitem{ECM93} D.~J.~Evans, E.~G.~D.~Cohen, and G.~P.~Morriss,
%{\it Probability of second law violations in shearing steady states},
Phys.~Rev.~Lett. {\bf 71}, 2401--2404 (1993).

\bibitem{Ga04} G.~Gallavotti, 
%{\it Entropy production in
%  nonequilibrium stationary states: a point of view},
Chaos, {\bf 14}, 680--690, (2004).

\bibitem{Ru04} D.~Ruelle, 
%{\it Conversations on nonequilibriun physics with an 
%extraterrestrial}, 
Physics Today, May, 48-53 (2004).

\bibitem{BGG97} F.~Bonetto, G.~Gallavotti, and P.~L.~Garrido, 
%{\it Chaotic principle: an experimental test},
Physica D {\bf 105}, 226--252 (1997).

\bibitem{BCL98} F.~Bonetto, N.~I.~Chernov, J.~L.~Lebowitz,
%{\it (Global and local) fluctuations of phase-space contraction
%in deterministic stationary non-equilibrium},
Chaos {\bf 8}, 823--833 (1998).

\bibitem{BPV98} L.~Biferale, D.~Pierotti, and A.~Vulpiani,
%{\it Time-reversible dynamical systems for turbulence},
J.~Phys.~A:~Math.~Gen. {\bf 31}, 21--32 (1998).

\bibitem{GP99} G.~Gallavotti, F.~Perroni,
%{\it An experimental test of the local fluctuation theorem in chains of
%weakly interacting Anosov systems},
chao-dyn/9909007.

\bibitem{GRS04} G.~Gallavotti, L.~Rondoni, and E.~Segre,
%{\it Lyapunov spectra and nonequilibrium ensembles equivalence in 2D fluid mechanics},
Physica D {\bf 187}, 338--357 (2004).

\bibitem{ZRA04a} F.~Zamponi, G.~Ruocco, L.~Angelani,
%{\it Fluctuations of entropy production in the isokinetic ensemble},
J.~Stat.~Phys. {\bf 115}, 1655--1668 (2004).

\bibitem{CL98} S.~Ciliberto and C.~Laroche,
%{\it An experimental test of the Gallavotti-Cohen fluctuation theorem},
J.~Phys.~IV {\bf 8}, Pr6-215 (1998).

\bibitem{GGK01} W.~I.~Goldburg, Y.~Y.~Goldschmidt, and H.~Kellay,
%{\it Fluctuation and Dissipation in Liquid-Crystal Electroconvection},
Phys.~Rev.~Lett. {\bf 87}, 245502 (2001).

\bibitem{CGHLP03}
S.~Ciliberto, N.~Garnier, S.~Hernandez, C.~Lacpatia, J.-F.~Pinton, 
and G.~Ruiz~Chavarria, 
%{\it Experimental test of the Gallavotti-Cohen fluctuation
%theorem in turbulent flows},
Physica A {\bf 340}, 240 (2004).

\bibitem{FM04}  K.~Feitosa and N.~Menon,
%{\it Fluidized Granular Medium as an Instance of the Fluctuation Theorem},
Phys.~Rev.~Lett. {\bf 92}, 164301 (2004).

\bibitem{GR97} G.~Gallavotti, D.~Ruelle, 
%{\it SRB states and nonequilibrium statistical mechanics close to
%  equilibrium}, 
Comm.~Math.~Phys. {\bf 190}, 279--285 (1997).

\bibitem{Ga96a} G.~Gallavotti, 
%{\it Extension of Onsager's reciprocity to large fields and 
%the chaotic hypothesis},
Phys.~Rev.~Lett. {\bf 77}, 4334--4337 (1996).

\bibitem{Ga96b} G.~Gallavotti,
%{\it Chaotic hypothesis: Onsager reciprocity and 
%fluctuation-dissipation theorem},
J.~Stat.~Phys. {\bf 84}, 899 (1996).

\bibitem{BG97} F.~Bonetto, G.~Gallavotti,
%{\it Reversibility, coarse graining and the chaoticity principle},
Comm.~Math.~Phys. {\bf 189}, 263--276, (1997).

\bibitem{Ga99b} G.~Gallavotti, 
%{\it New methods in nonequilibrium gases and fluids},
Open Systems and Information Dynamics {\bf 6}, 101--136 (1999).

\bibitem{GGZ05}
G.~Gallavotti, A.~Giuliani and F.~Zamponi,
%{\it Fluctuation relation beyond linear response theory}, 
J.~Stat.~Phys. {\bf 119}, 909 (2005).

\bibitem{GKP05} C.~Giardina, J.~Kurchan and L.~Peliti,
cond-mat/0511248.


%%%%%%%%%%%%%%%%%%%%%%%%%%%%%%%%%%%%%%%%%%%%%%%%%%%%%%%%%%%%%%%%%%%%%%%%%%%%%%%%%%%%%%%%%%%%%%%%%%%%%%%%%%
%%%%%%%%%%%%%%%%%%%%%%%%%%%%%%%%%%%%%%%%%%%%%%%%%%%%%%%%%%%%%%%%%%%%%%%%%%%%%%%%%%%%%%%%%%%%%%%%%%%%%%%%%%

\bibitem{Ku98} J.~Kurchan, 
%{\it Fluctuation theorem for stochastic dynamics},
J.~Phys.~A {\bf 31}, 3719--3729 (1998).

\bibitem{LS99} J.~L.~Lebowitz and H.~Spohn,
%{\it A Gallavotti--Cohen-Type Symmetry in the Large Deviation 
%Functional for Stochastic Dynamics},
J.~Stat.~Phys. {\bf 95}, 333--365 (1999).

\bibitem{SCM04}
G.~Semerjian, L.~F.~Cugliandolo, and A.~Montanari,
%{\it On the Stochastic Dynamics of Disordered Spin Models},
J.~Stat.~Phys {\bf 115}, 493--530 (2004).

\bibitem{CR04}
A.~Crisanti and F.~Ritort,
%{\it Intermittency of glassy relaxation and the emergence of a 
%non-equilibrium spontaneous measure in the aging regime},
Europhys.~Lett. {\bf 66}, 253--259 (2004).

\bibitem{Se98}
M.~Sellitto, 
%{\it Fluctuations of entropy production in driven glasses},
cond-mat/9809186.

\bibitem{Cu02}
L.~F.~Cugliandolo, 
{\it Dynamics of glassy systems},
in ``Slow relaxation and nonequilibrium dynamics in condensed matter'', 
Les Houches Session 77, J-L Barrat et al eds., Springer-Verlag, 2002.

\bibitem{AG04}
B~Abou, F~Gallet,
%{\it Probing an nonequilibrium Einstein relation in an aging colloidal glass},
Phys.~Rev.~Lett. {\bf 93}, 160603 (2004).

\bibitem{PM04}
N.~Pottier, A.~Mauger,
%{\it Anomalous diffusion of a particle in an aging medium},
Physica A {\bf 332}, 15--28 (2004).

\bibitem{Po04}
N.~Pottier,
%{\it Out of equilibrium generalized Stokes--Einstein relation: 
%determination of the effective temperature of an aging medium},
Physica~A {\bf 345}, 472 (2005).

\bibitem{CK99} L.~F.~Cugliandolo and J.~Kurchan, 
%{\it A scenario for the dynamics in the small entropy production limit},
Frontiers in Magnetism,
J.~Phys.~Soc.~Japan {\bf 69}, Suppl.~A, 247--256 (2000).

\bibitem{Gr03}
D.~G.~Grier, 
%{\it A revolution in optical manipulation},
Nature (London) {\bf 424}, 810--816 (2003).

\bibitem{WSMSE02}
G.~M.~Wang, E.~M.~Sevick, E.~Mittag, D.~J.~Searles, and D.~J.~Evans,
%{\it Experimental Demonstration of Violations of the Second Law of Thermodynamics for 
%Small Systems and Short Time Scales},
Phys.~Rev.~Lett. {\bf 89}, 050601 (2002).

\bibitem{CKZ?}
F.~Zamponi, F.~Bonetto, L.~F.~Cugliandolo, J.~Kurchan,
%{\it Fluctuation theorem for a driven Brownian particle coupled to a complex bath},
J. Stat. Mech. (2005) P09013.

\bibitem{ZRA04b}
F.~Zamponi, G.~Ruocco and L.~Angelani,
%{\it Generalized fluctuation relation and effective temperature in a driven fluid},
Phys.Rev.E {\bf 71}, 020101(R) (2005).


%%%%%%%%%%%%%%%%%%%%%%%%%%%%%%%%%%%%%%%%%%%%%%%%%%%%%%%%%%%%%%%%%%%%%%%%%%%%%%

\end{thebibliography}

\end{document}
