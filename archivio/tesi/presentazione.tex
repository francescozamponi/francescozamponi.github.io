\documentclass[a4paper,12pt]{article}
\usepackage[italian]{babel}
\usepackage[latin1]{inputenc}
\usepackage{amsmath,amsfonts,amssymb}
\begin{document} 

\noindent
\textbf{Francesco Zamponi}\\
\textbf{Universit\`a di Roma ``La Sapienza''}\\

\vspace{10mm}

\noindent
\textbf{\Large Un metodo per la misurazione delle temperature interne negli stati stazionari fuori dall'equilibrio: relazione illustrativa}\\

\vspace{10mm}

\noindent
\textbf{Obiettivi}
\vspace{3mm}\\
Questa tesi si inserisce all'interno di un articolato programma di ricerca, che si pone come obiettivo la caratterizzazione degli stati stazionari fuori dall'equilibrio di vari sistemi fisici: in particolare liquidi semplici e molecolari (acqua, glicerolo) e sistemi ``vetrosi'' (polimeri e colloidi). \\
Lo studio degli stati stazionari fuori dall'equilibrio, prodotti da forze termodinamiche dissipative agenti sul sistema accoppiato a un termostato, presenta un grande interesse di principio non essendo ancora disponibile una estensione dei principi della meccanica statistica alla descrizione di queste situazioni. La termodinamica e l'idrodinamica forniscono strumenti adeguati alla trattazione macroscopica di questi sistemi \cite{Mazur}, ma la costruzione di una trattazione statistica a partire dalla dinamica microscopica \`e resa difficile dal fatto che non esiste, in questo caso, un analogo dell'{\it ensemble} microcanonico, che viene introdotto, per sistemi all'equilibrio, sulla base dell'{\it ipotesi ergodica} e della struttura hamiltoniana delle equazioni dinamiche. La presenza di forze dissipative rende non hamiltoniana la dinamica del sistema, e la costruzione di misure invarianti \`e di conseguenza molto difficoltosa. Alcuni interessanti risultati ottenuti recentemente in ambito teorico \cite{Gallavotti&c} hanno generato un notevole interesse per la caratterizzazione sperimentale delle fluttuazioni statistiche intorno a stati stazionari fuori dall'equilibrio.\\
Un secondo motivo di interesse proviene dalla fisica dei sistemi ``vetrosi''; questi sistemi sono strutturalmente analoghi a dei liquidi, ma i tempi di rilassamento caratteristici del sistema all'equilibrio sono molto lunghi rispetto ai tempi tipici di un esperimento umano (qualche giorno), per cui il sistema si trova generalmente fuori equilibrio, ed evolve lentamente verso l'equilibrio, che viene per\'o raggiunto solo su scale di tempo astronomiche. In questi sistemi quindi {\it non sono presenti forze termodinamiche dissipative} e la dinamica \`e hamiltoniana, ma il sistema \`e fuori dall'equilibrio e {\it non \`e stazionario}. E' stato mostrato recentemente \cite{kurchan} che l'introduzione di forze dissipative esterne (sforzi di taglio, ad esempio inserendo il sistema tra due piatti in movimento relativo) ha l'effetto di rendere il sistema stazionario, ed \`e stato proposto un legame tra gli stati stazionari fuori equilibrio cos\`i ottenuti e gli stati termodinamici che il sistema attraversa durante l'evoluzione verso l'equilibrio in assenza delle forze esterne. In particolare si \`e osservato che le funzioni di correlazione di alcune osservabili sono molto simili nelle due situazioni, il che fornisce un'ulteriore motivazione per lo studio delle fluttuazioni intorno agli stati stazionari fuori dall'equilibrio indotti da forze esterne in questi sistemi.\\
Obiettivo della tesi \`e quindi la progettazione e la realizzazione di una cella per misure di diffusione della luce su campioni sottoposti a sforzi di taglio, i cui parametri caratteristici permettano di compiere misure sia su liquidi semplici che su sistemi vetrosi. Tale cella dovrebbe consentire la verifica di alcune previsioni teoriche e una migliore caratterizzazione degli stati stazionari fuori dall'equilibrio in questi sistemi.\vspace{3mm}\\
\textbf{Temperature effettive o interne}\vspace{3mm}\\
Nella descrizione del processo di {\it invecchiamento} nei sistemi vetrosi, ovvero della loro lentissima evoluzione verso l'equilibrio, \`e stato introdotto un parametro termodinamico detto {\it temperatura effettiva}. La temperatura effettiva \`e legata alla dinamica lenta di alcuni gradi di libert\`a del sistema, \`e diversa dalla temperatura del termostato se il sistema non \`e all'equilibrio, ed \`e detta ``temperatura'' perch\'e controlla gli scambi di calore tra il sistema e un termometro che abbia un tempo di risposta dell'ordine del tempo di rilassamento caratteristico del sistema. Recentemente \`e stata studiata \cite{kurchan2} la possibilit\`a di accoppiare al sistema fuori equilibrio un termometro la cui scala di tempo possa essere variata dall'esterno. Quando la scala di tempo del termometro \`e dell'ordine delle scale di tempo tipiche delle vibrazioni atomiche ($\sim 10^{-13}$ s) la temperatura letta dal termometro \`e quella del termostato; ma se il tempo di risposta del termometro viene reso confrontabile con i tempi di rilassamento caratteristici del sistema ($\sim 100$ s) il termostato legge una temperatura pi\`u alta, legata alla temperatura effettiva che caratterizza il sistema. Questo comportamento dovrebbe essere verificato sia durante il processo di invecchiamento, sia in presenza di forze esterne che rendano il sistema stazionario. \\
Il concetto di temperatura effettiva ha acquisito negli ultimi anni notevole importanza nella fisica dei sistemi vetrosi, ed \`e collegato con il comportamento delle fluttuazioni fuori dall'equilibrio: infatti le fluttuazioni sulle scale di tempo pi\`u lunghe risultano essere termalizzate alla temperatura effettiva e non alla temperatura del termostato. Quest'ultima affermazione deriva dallo studio delle relazioni di fluttuazione-dissipazione, ovvero della relazione tra funzione di correlazione e funzione di risposta della stessa osservabile, che all'equilibrio \`e di semplice proporzionalit\`a, con coefficiente l'inverso della temperatura. Fuori dall'equilibrio la relazione di fluttuazione-dissipazione pu\`o essere generalizzata e ha un ruolo molto importante nella definizione termodinamica del concetto di temperatura effettiva \cite{kurchan3}.\\
Risultano tuttora aperti dal punto di vista sperimentale i seguenti problemi:\\
{\it i)} Studio delle relazioni di fluttuazione-dissipazione fuori dall'equilibrio, per verificare la presenza di una temperatura effettiva ben definita operativamente. {\it ii)} Verifica della previsione teorica sul comportamento di un termometro al variare del suo tempo caratteristico di risposta in presenza di una temperatura effettiva diversa da quella del bagno termico. {\it iii)} Verifica della previsione teorica sull'analogia del concetto di temperatura effettiva durante l'invecchiamento o in stato stazionario indotto da forze esterne.\\
La cella realizzata durante questo lavoro di tesi permetter\`a di studiare le relazioni di fluttuazione-dissipazione utilizzando le fluttuazioni di densit\`a studiate mediante diffusione della luce.\vspace{3mm}\\
\textbf{Le simulazioni al calcolatore}\vspace{3mm}\\
Si \`e inoltre deciso di affiancare al lavoro sperimentale delle simulazioni al calcolatore, riguardanti la verifica delle previsioni ottenute in \cite{kurchan2} sul comportamento di un termometro al variare del tempo di risposta caratteristico. Si \`e dunque accoppiato a un sistema modello di vetro (una miscela binaria di ``sfere soffici'' con potenziale di interazione $\propto 1/r^{12}$) un modello di termometro, costituito da un oscillatore armonico di frequenza variabile. Il sistema era accoppiato a un termostato e sottoposto a uno sforzo di taglio costante utilizzando l'algoritmo SLLOD \cite{Evans}, che utilizza delle equazioni del moto non hamiltoniane per introdurre la dissipazione dovuta alla forza esterna. Il programma \`e stato scritto interamente durante il lavoro di tesi in linguaggio {\it C++ object oriented}, in modo da verificare i vantaggi e gli svantaggi introdotti da questa tecnica di programmazione rispetto a un normale programma in Fortran.\\
Si \`e quindi misurata l'energia media dell'oscillatore al variare della sua frequenza; il risultato \`e in accordo con le previsioni teoriche, dal momento che ad alta frequenza l'oscillatore legge la temperatura del bagno termico mentre a bassa frequenza la sua energia media \`e uguale entro gli errori alla temperatura effettiva misurata tramite le relazioni di fluttuazione-dissipazione.\\
Questo tipo di esperimento condotto al calcolatore potr\`a essere ripetuto su un sistema fisico utilizzando la cella realizzata durante questo lavoro di tesi; infatti \`e possibile con tecniche ottiche confinare particelle di massa variabile all'interno del campione sottoponendole a un potenziale armonico. Variando la massa della particella si pu\`o variare la frequenza caratteristica delle oscillazioni nella buca di potenziale, riproducendo quindi sperimentalmente le condizioni della simulazione al calcolatore.\vspace{3mm}\\
\textbf{Struttura}\vspace{3mm}\\
La struttura della tesi \`e la seguente: \\
Nel capitolo 1 viene presentata una rassegna dei risultati pi� recenti che riguardano la definizione della temperatura effettiva; si cerca in particolare di mettere in luce le previsioni che si ottengono a partire dalle varie teorie. Si discute prima l'introduzione della temperatura effettiva durante l'invecchiamento, e  poi l'estensione di questo concetto agli stati stazionari fuori equilibrio.\\
Nel capitolo 2 si discute l'interpretazione della temperatura effettiva come parametro che controlla gli scambi di calore del sistema con termometri ``lenti''; si vede che questa propriet� � una utile definizione operativa di temperatura fuori dall'equilibrio. Si discutono poi alcuni possibili metodi alternativi per la sua misurazione.\\
Nel capitolo 3 vengono presentati i risultati ottenuti dalle simulazioni al calcolatore: in particolare si verifica il corretto funzionamento dell'oscillatore armonico come termometro.\\
Nel capitolo 4 si discute il progetto della cella e alcune possibilit� per il suo utilizzo; vengono poi presentati i risultati preliminari.\vspace{3mm}\\
\textbf{Risultati ottenuti e conclusioni}\vspace{3mm}\\
Il presente lavoro di tesi si propone quindi la verifica di alcuni risultati ottenuti recentemente in ambito teorico riguardo alle fluttuazioni in stati stazionari fuori dall'equilibrio. Gli obiettivi principali sono: {\it i)} Studio delle funzioni di correlazione delle fluttuazioni di densit\`a mediante diffusione della luce. {\it ii)} Studio delle relazioni di fluttuazione-dissipazione utilizzando ancora le fluttuazioni di densit\`a. {\it iii)} Misurazione della temperatura effettiva sia attraverso le relazioni di fluttuazione-dissipazione che utilizzando un oscillatore armonico di frequenza variabile come termometro.\\
Per far questo \`e stata realizzata una cella per misure di diffusione della luce in grado di sottoporre il campione a uno sforzo di taglio variabile, in un intervallo interessante per lo studio sia di liquidi semplici che di sistemi vetrosi. Inoltre \`e stato compiuto uno studio numerico preliminare per verificare la correttezza delle previsioni riguardanti la possibilit\`a sperimentale di utilizzare un oscillatore armonico per misurare la temperatura effettiva. \\
Entrambi gli studi hanno avuto esito positivo, e attualmente \`e in corso l'installazione nel laboratorio dell'apparato di fotocorrelazione che sar\`a utilizzato per iniziare i primi cicli di misure.


\begin{thebibliography}{99}

\bibitem{Mazur} S. R. De Groot, P. Mazur, {\it Non-equilibrium thermodynamics}, Dover
\bibitem{Gallavotti&c} G.Gallavotti, E.G.D.Cohen, Phys. Rev. Lett. 74, 2694 (1995); F.Bonetto, G.Gallavotti, P.Garrido, Physica D 105, 226 (1997); L. Bertini, A. De Sole, G. Gabrielli, G. Jona-Lasinio, C. Landim, Phys. Rev. Lett. 87, 040601-1 (2001).
\bibitem{kurchan} L.Berthier, J.-L.Barrat, J.Kurchan, Phys. Rev. E 61, 5464 (2000)
\bibitem{kurchan2} J.Kurchan, L.Cugliandolo, L.Peliti, Phys. Rev. E 55, 3898 (1997)
\bibitem{kurchan3} L.F.Cugliandolo, J.Kurchan, Phys. Rev. Lett. 71, 173 (1993); Philosophical Magazine 71, 501 (1995); J. Phys. A: Math. Gen. 27, 5749 (1994)
\bibitem{Evans} D.J.Evans, G.P.Morris, {\it Statistical mechanics of nonequilibrium fluids}, Academic Press

\end{thebibliography}




\end{document}




