\documentclass[aps,pre]{revtex4}
\usepackage[italian]{babel}
\usepackage[latin1]{inputenc}
\usepackage{amsmath,amsfonts}
\newcommand{\blankline}{\vskip .3cm}
\newcommand{\beq}{\begin{equation}}
\newcommand{\eeq}{\end{equation}}
% This fixes the margins and page sizes
%\setlength{\hoffset}{0pt} \setlength{\voffset}{0pt} \setlength{\topmargin}{-20pt}
%\setlength{\headsep}{30pt} \addtolength{\headsep}{-\headheight}
%\setlength{\textheight}{9in} \addtolength{\textheight}{-40pt}
%\setlength{\footskip}{30pt} \setlength{\oddsidemargin}{0pt}
%\setlength{\textwidth}{6.5in}
\begin{document}

\title{Grandi fluttuazioni nel caso di variabili indipendenti} 

\author{Francesco Zamponi} 

\affiliation{Dipartimento di Fisica dell'Universit� di Roma ``La Sapienza''} 


\maketitle

\section{Teorema del limite centrale}

\noindent
Consideriamo $N$ variabili $y_i$ indipendenti con distribuzione $p(y_i)$.
Vogliamo calcolare la distribuzione della variabile $l=N^{-1} \sum_i y_i$ nel
limite $N \rightarrow \infty$.
Scriviamo, definendo $dp(y_i)=p(y_i)dy_i$,
\beq
p(l)=\int \prod_i dp(y_i) \ \delta \left( \sum_i y_i - N l \right) = \int \prod_i dp(y_i) \int d\lambda e^{\lambda \sum_i y_i - N \lambda l}
\eeq
usando la rappresentazione integrale della funzione $\delta$ e ruotando il cammino di integrazione sull'asse immaginario del piano $\lambda$ complesso.
Definiamo
\beq
Z(\lambda) = \int dp(y) e^{\lambda y} = e^{f(\lambda)}
\eeq
E' chiaro che $f(0)=0$, $f'(0)=\langle y \rangle=m$, $f''(0)=\langle (y-\langle y \rangle)^2 \rangle = \sigma^2$.
Otteniamo, per $N \rightarrow \infty$,
\beq
\label{sella}
p(l) = \int d\lambda e^{-N (\lambda l - f(\lambda))} \sim e^{-N (\bar{\lambda} l - f(\bar{\lambda}))} 
\eeq
dove $\bar{\lambda}$ \`e definito da $l= f'(\bar{\lambda})$. Assumendo che $\bar{\lambda}$ sia piccolo otteniamo
\beq
l \sim f'(0) + f''(0) \bar{\lambda} = m + \sigma^2 \bar{\lambda} \hspace{1cm} \Rightarrow  \hspace{1cm} \bar{\lambda} = \frac{l-m}{\sigma^2}
\eeq
e quindi sviluppando nella (\ref{sella}) $f(\lambda) \sim m \lambda + \frac{1}{2} \sigma^2 \lambda$ si ottiene
\beq
p(l) \sim e^{ - N \frac{(l-m)^2}{2\sigma^2}}
\eeq
Quindi la variabile $l$ ha distribuzione Gaussiana con valor medio $m$ e deviazione standard $\sigma/\sqrt{N}$. Questo risultato \`e valido con la condizione
\beq
\bar{\lambda} \ll \frac{\sigma}{\sqrt{N}}
\eeq
che assicura che $\bar{\lambda}$ sia molto piccolo nel limite di $N \rightarrow \infty$. 

\section{Grandi deviazioni}

\noindent
Una grande deviazione si ha se ad esempio $|l - m| > a$ dove $a$ \`e una costante fissata. Infatti abbiamo visto che le fluttuazioni tipiche di $l$ rispetto al suo valor medio sono dell'ordine di $1/\sqrt{N}$ e quindi tendono a $0$ nel limite $N \rightarrow \infty$. Stimiamo quindi la probabilit\`a di una grande fluttuazione con $l - m > a$:
\beq
p(l-m > a) = \int_{l-m>a} \prod_i dp(y_i) \leq  \int_{l-m>a} \prod_i dp(y_i) e^{N \lambda (l-m-a)} \leq \int \prod_i dp(y_i) e^{N \lambda (l-m-a)}
\eeq
dove la prima maggiorazione \`e valida per ogni $\lambda$ dal momento che l'integrale \`e sulla regione $l-m>a$ e la seconda maggiorazione \`e dovuta al fatto che l'integrando \`e sempre positivo.
Sostituiamo $l = N^{-1} \sum_i y_i$ e otteniamo
\beq
p(l-m>a) \leq e^{N (f(\lambda) - \lambda f'(0) - \lambda a)}
\eeq
Ora usiamo lo sviluppo di Taylor $f(\lambda) = f'(0) \lambda + \frac{1}{2} f''(\tilde{\lambda}) \lambda^2$ con $\tilde{\lambda} \in [0,\lambda]$ e otteniamo
\beq
p(l-m>a) \leq e^{N (\frac{1}{2} f''(\tilde{\lambda}) \lambda^2 - \lambda a)}
\eeq
Se ora assumiamo che $f''(\tilde{\lambda}) < C$ per $\lambda \in [0,\lambda_{max}]$, possiamo maggiorare ancora con
\beq
p(l-m>a) \leq e^{N (\frac{1}{2} C \lambda^2 - \lambda a)}
\eeq
e dal momento che la stima \`e valida per ogni $\lambda<\lambda_{max}$ possiamo prendere il minimo su $\lambda$ ottenendo
\beq
p(l-m>a) \leq e^{-N \frac{a^2}{2C}}
\eeq
purch\`e $a/C \in [0,\lambda_{max}]$. La verifica di queste ipotesi dipende essenzialmente dal fatto che la distribuzione $p(y)e^{\lambda y}$ abbia varianza finita per ogni $\lambda$, anche se crescente con $\lambda$. La stima \`e analoga nel caso $l-m < -a$.



%\begin{thebibliography}{99}

%\bibitem{GMM} G.Gallavotti, A.Martin-Lof, S.Miracle-Sol\'e, Lecture Notes in Physics, vol. 20, ed. A.Lenard, Springer-Verlag, 1973, Heidelberg

%\end{thebibliography}

\end{document}
