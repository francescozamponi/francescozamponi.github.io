\documentclass[a4paper,10pt]{article}

%\usepackage[italian]{babel}
%\usepackage[latin1]{inputenc}
\usepackage{amsmath,amsfonts}
\usepackage{hyperref,color}
\newcommand{\bl}{\vskip .3cm}
\newcommand{\f}{\begin{equation}}
\newcommand{\ff}{\end{equation}}

% This fixes the margins and page sizes
\setlength{\hoffset}{0pt} \setlength{\voffset}{0pt} \setlength{\topmargin}{-20pt}
\setlength{\headsep}{30pt} \addtolength{\headsep}{-\headheight}
\setlength{\textheight}{10in} \addtolength{\textheight}{-40pt}
\setlength{\footskip}{30pt} \setlength{\oddsidemargin}{0pt}
\setlength{\textwidth}{6.5in}

\begin{document}

%\pagenumbering{void}

\centerline{\LARGE \bf Curriculum vitae, Francesco Zamponi}

\noindent
\begin{table}[h]
\begin{tabular}{ll}
{\bf Date and place of birth}  & Rome (Italy), February 23, 1979 \\
{\bf Professional address} & Dipartimento di Fisica, Sapienza Universit\`a di Roma, \\
& P.le A.Moro 2, 00185 Rome, Italy \\
{\bf phone} & +39-06-4991-4288 \\
{\bf e-mail} & name.surname at uniroma1.it \\
{\bf WWW} & \href{https://www2.phys.uniroma1.it/doc/zamponi}{{\color{blue} https://www2.phys.uniroma1.it/doc/zamponi}} \\
{\bf ORCID} & 0000-0001-9260-1951 \\ 
\end{tabular}
\end{table}

\noindent {\large \bf Education and Positions Held}
\begin{table}[h]
\begin{tabular}{ll}

{\bf Positions:} & \\
09/2023 - present & {Professore associato}, Dipartimento di Fisica, Sapienza Universit\`a di Roma \\
10/2018 - 08/2023 &  {\it Directeur de Recherche (DR2)}, CNRS, LPENS, {\it Ecole Normale Sup\'erieure}, Paris \\
10/2018 - 08/2021 & {\it Professeur associ\'e}, Physics Department, {\it Ecole Normale Sup\'erieure}, Paris \\
09/2015 - 09/2018 & {\it Ma\^itre de Conf\'erences associ\'e}, Physics Department, {\it Ecole Normale Sup\'erieure}, Paris \\
03/2012 - 09/2018 &  {\it Charg\'e de Recherche (CR1)}, CNRS, LPT, {\it Ecole Normale Sup\'erieure}, Paris \\
03/2008 - 02/2012 & {\it Charg\'e de Recherche (CR2)}, CNRS, LPT, {\it Ecole Normale Sup\'erieure}, Paris \\
10/2007 - 02/2008 & EU Marie Curie Fellowship, LPT, {\it Ecole Normale Sup\'erieure}, Paris \\
10/2006 - 10/2007 & Post Doc, SPhT Saclay, Paris \\
10/2005 - 10/2006  &
Post Doc, LPT, {\it Ecole Normale Superi\'eure}, Paris \\
11/2002 - 10/2005  & 
PhD student, Physics Department, University of Rome {\it La Sapienza}; \\
 & Supervisors: Prof. Giancarlo Ruocco and Prof. Giorgio Parisi \\
01/2002 - 10/2002 & INFM fellowship, Rome {\it La Sapienza} \\
09/1997 - 12/2001 & Undergraduate studies, University of Rome {\it La Sapienza} \\
\\
{\bf Qualifications:} & \\
2023 & Abilitazione Scientifica Nazionale (fascia I), settore 02/D1 (scad.11/1/2035) \\
2013 & Abilitazione Scientifica Nazionale (fascia I), settore 02/A2 (scad.14/4/2033)  \\
2013 & Abilitazione Scientifica Nazionale (fascia I), settore 02/B2 (scad.11/5/2033)  \\
30/11/2012 & {\it Habilitation \`a diriger des recherches}, {\it Ecole Normale Sup\'erieure}, Paris \\
13/01/2006 & PhD in Physics, Universit\`a di Roma {\it La Sapienza} \\
13/12/2001 & {\it Laurea in Fisica} (cum laude), Universit\`a di Roma {\it La Sapienza} \\
\end{tabular}
\end{table}


\noindent {\large \bf Awards, grants, fellowships, long-term visiting positions} \\
\begin{table}[h]
\begin{tabular}{ll}
2002 & INFM--{\it La Sapienza} prize 
for an experimental {\it Laurea} thesis discussed in the year 2001.\\
2007 & ``Giorgio Gamberini'' award of ``Scuola Normale Superiore'', Pisa, for a PhD Thesis \\
 & in theoretical physics discussed after January 2005. \\
2009 & Visiting fellow, Princeton Center for Theoretical Science (PCTS), Princeton, NJ
(09-12/2009) \\
2009 & BQR grant of ENS: {\it Fluids and solids at low temperature} \\
2010 & MIT-France Seed Fund/MISTI Global Seed Fund grant: {\it Numerical simulations and quantum
algorithms} \\
2011 & PIR grant of ENS: {\it Optimization in complex systems} \\
2013 & Appointment as a regular visiting fellow, Duke University, Durham, NC \\
2014 & Grant of the university PSL$^\star$ for the project {\it Jamming: Theory and Experiment} with O.Dauchot \\ 
2016 & Member of the Simons collaboration {\it Cracking the glass problem} (\href{http://scglass.uchicago.edu}{{\color{blue} website}})  \\
2016 & ERC Consolidator grant {\it GlassUniversality} \\
\end{tabular}
\end{table}




\newpage

\noindent {\LARGE \bf \underline{Scientific committees and other responsibilities}}

\begin{table}[h]
\begin{tabular}{ll}
2010-14 & Member of the scientific committee of the GDR Phenix-CNRS \\
2012-16 & Elected member of the {\it Comit\'e national de la Section 02} of CNRS \\
2013 & Member of the AERES visiting committee of the LPT-Orsay \\
2014 & Member of the selection committee for a {\it Ma\^itre de Conf\'erences} position at LPT-Orsay \\
2015-16 & Elected member of the {\it Comission interdisciplinaire 51} of CNRS \\
2019-21 & Elected as head of the {\it Statistical physics} group of LPENS, Paris \\
\end{tabular}
\end{table}

\vskip15pt


\noindent
{\LARGE \bf \underline{Research}}

\vskip15pt
\noindent {\large \bf Research interests} 
see \url{http://www.phys.ens.fr/~zamponi/pagine/research.html}


\vskip15pt
\noindent {\large \bf Publications} 
see \url{http://www.phys.ens.fr/~zamponi/pagine/pubblicazioni.html}

\vskip15pt
\noindent
{\large \bf Talks}
see \url{http://www.phys.ens.fr/~zamponi/pagine/talks.html}

\vskip15pt

\noindent {\large \bf Software}
see \url{http://www.phys.ens.fr/~zamponi/pagine/software.html}

\vskip15pt

\noindent {\large \bf Organization of scientific events}
see \url{http://www.phys.ens.fr/~zamponi/pagine/events.html}

\vskip15pt
\noindent {\large \bf Referee} for:  
European Journal of Physics B, Europhysics Letters,
Journal of Chemical Physics,  Journal of Physics: Condensed Matter, 
Journal of Statistical Mechanics: Theory and Experiment (JSTAT), Journal of Statistical Physics,  
Nature, Nature Physics, Nature Communications,
Journal of Alloys and Compounds, Powder Technology,
Physica A,  Physical Review E, Physical Review Letters, Molecular Physics, Soft Matter;
see \href{https://orcid.org/0000-0001-9260-1951}{{\color{blue}ORCID}}.
\\


\vskip15pt

\noindent {\large \bf Short-term visits -- Year[number of days]} \\
\begin{table}[h]
\begin{tabular}{ll}
Duke University &   2013[8]  \\
Graduate center, CUNY  & 2014[3] \\
MIT Boston &   2011[8], 2010[11]   \\
Princeton University & 2009[107] \\
Porto Alegre University &  2011[11], 2010[11]    \\
Rome ``Sapienza'' &  2014[24], 2013[15], 2012[17], 2011[30], 2011[7] \\
SISSA Trieste & 2010[6], 2010[12], 2009[18], 2008[16] \\
Weizmann Institute & 2013[5]    \\
\end{tabular}
\end{table}






\newpage

\noindent {\LARGE \bf \underline{Teaching}}

\vskip10pt
\noindent {\bf Courses (AY = Academic Year)}
\begin{itemize}
\item {\it Metodi matematici della fisica} 
and {\it Metodi e modelli matematici della fisica} \\
Undergraduate courses of the
Department of Physics, 
University of Rome {\it La Sapienza} \\
AY2003-04, 2$\times$20 hours, assistant to A.~Degasperis \\
Lecture notes, with exercises, published on the WWW site of the Department
\item {\it Fisica dei liquidi} \\
Undergraduate course of the
Department of Physics, 
University of Rome {\it La Sapienza} \\
AY2004-05 and AY2006-07, 5 hours, assistant to P.~Tartaglia \\
The lecture notes can be downloaded from my web page
\item {\it M\'ecanique statistique (preceptorat)} \\
Undergraduate course (L3) of the ESPCI, Paris \\
AY2008-09 $\to$ AY2011-12, 16 hours, assistant to J.-F.~Joanny  
\item {\it Spin glasses} \\
$\star$ AY2007-08 $\to$ AY2009-10, 20 hours, PhD in Statistical Mechanics, SISSA, Trieste \\
$\star$ AY2010-11 $\to$ AY2013-14, 20 hours, PhD in Physics, Rome {\it La Sapienza} \\
$\star$ 2010, 3 hours, Winter School ``Understanding and exploiting complexity at the nanoscale'', \\
\phantom{$\star$ 2010,} Universit\'e libre de Bruxelles, Belgium \\
$\star$ 2011, 3 hours, University of Porto Alegre, Brazil \\
The lecture notes can be downloaded from my web page or on {\tt arXiv:1008.4844}
\item {\it Introduction \`a la Physique Num\'erique} \\
AY2011-12 $\to$ AY2015-16, 12 hours, with V.~Croquette,
course L3 of the ENS, Paris  
\item {\it Machine learning}, a course at the private company ``CFM'', Paris 2017 \\
Together with F.~Krzakala and A.~Manoel, 6 hours.
\item {\it Math\'ematiques pour physiciens} \\
AY2015-16$\to$2018-19, 38 hours, with M.~Petrini,
course L3 of the ENS, Paris  
\item {\it Information, inference, networks: from statistical physics to big biological data} \\
AY2016-17$\to$2020-21, 20 hours, with R.~Monasson and S.~Cocco, 
ICFP master (M2), ENS Paris
\item {\it Statistical Physics 2: Disordered Systems and Interdisciplinary Applications} \\
AY2020-21$\to$AY2022-23, 20 hours, with G.~Schehr, ICFP master (M2), ENS Paris
\item {\it Meccanica analitica e relativistica} \\
Undergraduate course of the Department of Physics, 
{\it Sapienza} University of Rome  \\
AY2023-24$\to$ present, 60 hours
\item {\it Modelli matematici per la fisica II} \\
Undergraduate course of the Department of Mathematics, 
{\it Sapienza} University of Rome  \\
AY2023-24$\to$ present, 56 hours



\end{itemize}

\clearpage

\vskip10pt
\noindent {\bf Undergraduate students}

\begin{table}[h]
\begin{tabular}{lllll}
Year  & Student   & Stage   & Institute & Co-supervisor \\
\hline
2004 & Alessio Andronico & Laurea  & Sapienza Rome & G.~Ruocco \\
2005 & Francesco Caltagirone & Dissertazione  & Sapienza Rome&  G.~Ruocco \\
2008 & Francesco Caltagirone & Laurea  & Sapienza Rome&  G.~Parisi \\
2008 & Ousmane Kodio  & L3   & Magist\`ere de Physique d'Orsay  & ------ \\
2009 & Indaco Biazzo     & Laurea  & Sapienza Rome  & G.~Parisi \\
2009 & Hugo Jacquin   & M2   & Master ``Syst\`emes Complexes'' & ------ \\
2010 & Emmanuelle Kapulumba & L3  & Universit\'e Paris 7  & ------ \\
2011 & Nicolas Allegra   & M2 & Master ``Syst\`emes Complexes'' & A.~Walczak \\
2011 & Antonie Ferm\'e  & L3 & ENS Paris & T.~Bodineau \\
2011 & Vishal Agarwal &  Internship & IIT Guwahati, India & ------ \\
2013 & Giulia Carra    & M2   & ENS Paris   & P.~Charbonneau \\
2014 & Giulia Cencetti & M2  & ENS Paris   & ------ \\
2014 & Marcus Cordi   & M2  & KTH Stockholm  & J.-P.~Bouchaud \\
2015 & Matthieu Mangeat & M2  & ENS Paris   & ------ \\
2015 & Giancarlo Croce   & M1 & ENS Paris  & S.~Cocco \\
2015 & Alexis Poncet     & M1  & ENS Paris   & E.~Corwin \\
2017 & Dhruv Sharma & M2  & ENS Paris & J.-P.~Bouchaud \\
2017 & Steffen Arnold  & Internship & ENS Paris & ------ \\
2017 & Guillaume Nevo & M2 & ENS Paris & J.-P.~Bouchaud \\
2020 & Jeanne Trinquier & M2 & Universit\'e de Paris & M.~Weigt \\
2020 & Matteo Bisardi & M2 & Universit\'e de Paris & M.~Weigt \\
2023 & Leonardo Di Bari & M2 & Sorbonne Universit\'e & M.~Weigt\\
2024 & Guglielmo Mennella & M2 & Sapienza University & P.~Charbonneau\\
2024 & Gabriele Farn\'e & M2 & Sapienza University & ------\\
2024 & Shoichi Yip & M2 & Sapienza University & C.~Nizak, M.~Weigt \\
2024 & Ottavia Schulte & M2 & Sapienza University & ------ \\
\end{tabular}
\end{table}

\newpage
%\vskip5pt

\noindent {\bf PhD students}

\begin{table}[h]
\begin{tabular}{lllll}
Years  & Student   &  Institute & Co-supervisor & Currently \\
\hline
2009-11 & Laura Foini & SISSA Trieste & A.~Gambassi & Permanent, CNRS - IPhT Paris \\
2010-13 & Victor Bapst & ENS Paris & G.~Semerjian  & Google DeepMind, London \\
2011-12 & Laetitia Papaxanthos & ENS Paris & J.-P.~Bouchaud & (PhD at ENS not completed) \\
2012-15 & Corrado Rainone & Sapienza Rome & G.~Parisi  & Qualcomm, Amsterdam\\
2013-16 & Thibaud Maimbourg & ENS Paris & J.~Kurchan & Postdoc, LPTMS Paris  \\
2016-19 & Camille Scalliet & Univ.~Montpellier & L.~Berthier & Permanent, CNRS - LPENS Paris\\
2017-20 & Dhruv Sharma & ENS Paris & J.-P.~Bouchaud & Talagent Financial, Boulder, CO \\
2020-23 & Jeanne Trinquier & Sorbonne Université & M.~Weigt & Postdoc, Institut de la vision, Paris \\
2020-23 & Matteo Bisardi & ENS Paris & M.~Weigt & Postdoc, Univ. of British Columbia \\
2020-23 & Enrico Ventura & Sapienza Rome & G.~Ruocco & Postdoc, Bocconi University, Milan\\
2023-26 & Lorenzo Rosset & ENS Paris & M.~Weigt & \\
2023-26 & Alya Zeinaty & Sorbonne Universit\'e & M.~Weigt & \\
2023-26 & Luca del Bono & Sapienza University & F.~Ricci-Tersenghi & \\
\hline
\end{tabular}
%\begin{tabular}{ll}
%2012 & Referee for the PhD thesis of Alessandro Sellerio, EPFL, Lausanne \\
%2012 & Referee for the PhD thesis of Michele Castellana, University of Rome ``Sapienza'' \\
%2012 & Referee for the PhD thesis of Xiaoquan Yu, SISSA, Trieste \\
%2020 & Referee for the PhD thesis of Wencheng Ji, EPFL, Lausanne \\
%2022 & Referee for the PhD thesis of Th\'eo Dessertaine, Ecole Polytechnique, Paris \\
%2022 & President of the Jury for the PhD thesis of Cyril Malbranke, ENS, Paris \\
%2023 & Referee for the PhD thesis of Luca Sesta, Politecnico di Torino \\
%2023 & Referee for the PhD thesis of Giovanni Piccioli, EPFL, Lausanne \\
%\end{tabular}
\end{table}

\noindent {\bf Post-docs}

\begin{table}[h]
\begin{tabular}{llll}
Years & Post-doc & Funding  & Currently\\
\hline
2012-14 & Yuliang Jin & PSL grant  & Permanent, ITP-CAS, Shangai \\
2012-14 & Stanislao Gualdi &  Crisis Project & Capital Fund Management, Paris \\
2015-18 & Beatriz Seoane Bartolom\'e & Marie Curie fellowship, ERC & Tenure track, UCM, Madrid \\
2016-18 & Elisabeth Agoritsas & Simons, ERC & Independent postdoc, EPFL Lausanne \\
2017-18 & Ada Altieri & ERC & Ma\^itre de Conf\'erences, Universit\'e de Paris \\
2017-20 & Harukuni Ikeda & Japanese grant, Simons & Associate Professor, Tokyo University \\
2017-19 & Dmytro Khomenko & Simons & Postdoc, Sapienza University \\
2018-19 & Anna Paola Muntoni & Simons & Independent postdoc, Politecnico di Torino \\
2018-21 & Alessandro Manacorda & ERC & Postdoc, Sapienza University \\
2019 & Davide Facoetti & ERC & Research scientist, Faculty AI  \\
2020 & Misaki Ozawa & ERC & Permanent, CNRS - LiPhy Grenoble \\
2020-22 & Giampaolo Folena & Simons, ERC & Postdoc, Duke University \\
2020-23 & Felix Mocanu & Simons, ERC & Postdoc, University of Oxford \\
2020-23 & Simone Ciarella & Simons, ERC & Permanent, Netherlands eScience Center\\
2021-22 & Sabrina Cotogno & Simons & Researcher, Capgemini Engineering \\
2023-26 & Saverio Rossi & Overheads & \\
\end{tabular}
\end{table}







\newpage

\noindent {\LARGE \bf \underline{Education} }\\

\vskip-5pt

\noindent {\bf Courses attended during the undergraduate and PhD studies} \\
\indent(Physics department of the University of Rome ``La Sapienza'') \\
\vskip-10pt
\begin{table}[h]
\begin{tabular}{ll}
1997-2000 & Standard courses of the first three years of the undergraduate course in Physics\\
2000-2001 & {\it Meccanica statistica} (Statistical Mechanichs), Prof.~C.~Di~Castro, 120h\\
& {\it Fisica dei sistemi disordinati} (Physics of Disordered Systems), Prof.~E.~Marinari, 120h\\
& {\it Fisica teorica III} (Quantum ElectroDynamics), Prof.~N.~Cabibbo, 120h\\
& {\it Fisica dei solidi} (Solid State Physics), Prof.~L.~Pietronero, 120h\\
& {\it Fisica delle basse temperature} (Quantum Many-Body Systems), Prof.~C.~Castellani, 120h\\
& {\it Programmazione orientata agli oggetti} (Object-oriented Programming), Prof.~G.~Organtini,
30h \\
2001-2002 & {\it Metodi matematici della fisica (corso avanzato)} (Stochastic Processes),
Prof.~G.~Jona-Lasinio, 120h \\
& {\it Meccanica superiore} (Statistical Mechanics, Dynamical Systems), 
Prof.~G.~Gallavotti, 120h \\
2002-2003 & {\it Introduzione al modello standard e alle sue estensioni supersimmetriche} \\
& \hspace{10pt} 
(An introduction to the Standard Model and to its Supersymmetric Extensions), \\
&\hspace{15pt}Dr.~E.~Franco
and Dr.~L.~Silvestrini, 60h \\
& {\it Fisica dei sistemi disordinati} (Physics of Disordered Systems - for PhD students), \\
&\hspace{15pt}Dr.~A.~Cavagna and Dr.~F.~Ricci-Tersenghi, 60h \\
& {\it Teoria dei campi} (Field Theory), Prof.~M.~Testa, 80h \\
\end{tabular}
\end{table}

\vskip-10pt
\noindent {\bf Schools and conferences}
\begin{itemize}
\item ``8th International Workshop on Disordered Systems'', Andalo (Trento), 
Italy, March~12--15, 2001.
\item ``Unifying Concepts in Glass Physics'', conference, {\it Accademia dei Lincei},
Roma, Italy, February~26--March~2, 2002.
\item ``Field Theory and Statistical Mechanics'', 
conference, {\it Dipartimento di Fisica dell'universit\`a ``La Sapienza''}, Roma, Italy,
June~10--15, 2002.
\item ``Spectroscopic investigation of the collective dynamics in disordered systems'', 
summer school, {\it International Center for Theoretical Physics ``Abdus Salam''} (ICTP), Trieste,
Italy, June~17--28, 2002.
\item ``Slow Relaxation and Nonequilibrium Dynamics in Condensed Matter'',
summer school, {\it \'Ecole de Physique des Houches}, Les Houches, France, July 1--26, 2002.
\item ``Perspectives in Mathematical Physics'', conference, 
{\it Dipartimento di Fisica dell'universit\`a ``La Sapienza''}, Roma, Italy,
September 4--7, 2002.
\item ``9th International Workshop on Disordered Systems'', Molveno (Trento), Italy, 
March 10--13, 2003.
\item ``The physics of complex systems (new advances and perspectives)'', summer school,
International School of Physics ``Enrico Fermi'', Course CLV, Varenna, Italy, 
July 1--11, 2003.
\item ``Supersimmetry and matrix models'',
{\it XII national seminar in Theoretical Physics}, INFN, Parma, Italy, 
September 1--12, 2003.
\item ``Applications of random matrices to physics'', 
Marie Curie Training Course, {\it \'Ecole de Physique des Houches}, 
Les Houches, France, June 6--25, 2004.
\item ``First Latin-American School and Conference on: Statistical Mechanics and Interdisciplinary Applications'', 
La Havana, Cuba, February, 28--March, 12, 2005.
\item ``Recent progress in glassy physics'', workshop,
Insitut Henri Poincar\'e, Paris, France, September 27--30, 2005.
\item ``Jamming'', workshop,
Aspen Center for Physics, Aspen (CO), USA, July, 29--August, 13, 2007.
\end{itemize}


\end{document}







